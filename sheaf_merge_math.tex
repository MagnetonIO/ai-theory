\documentclass{article}
\usepackage{amsmath,amssymb,amsthm} % Common math packages

\begin{document}

\section*{The Sheaf Condition: Mathematical Explanation}

Let $X$ be a topological space. A \emph{presheaf} $\mathcal{F}$ of sets (or groups, rings, etc.) on $X$ assigns:
\begin{itemize}
\item to each open set $U \subseteq X$, a set $\mathcal{F}(U)$ (called the sections of $\mathcal{F}$ over $U$),
\item to each inclusion of open sets $V \subseteq U$, a restriction map
\[
\rho_{V,U} : \mathcal{F}(U) \; \longrightarrow \; \mathcal{F}(V),
\]
such that $\rho_{W,V} \circ \rho_{V,U} = \rho_{W,U}$ whenever $W \subseteq V \subseteq U$ are open in $X$.
\end{itemize}

\medskip

A \emph{sheaf} is a presheaf $\mathcal{F}$ that satisfies the following additional \textbf{sheaf condition} (often stated in two parts):

\begin{enumerate}
\item \textbf{Local identity:} If $U \subseteq X$ is open and $\{U_i\}_{i \in I}$ is an open cover of $U$, then for any two sections $s,t \in \mathcal{F}(U)$, if 
\[
s \big|_{U_i} \;=\; t \big|_{U_i} 
\quad\text{for all } i \in I,
\]
then $s = t$. In other words, a section is determined by its restrictions to an open cover.

\item \textbf{Gluing (local data give a global section):} If $U \subseteq X$ is open and $\{U_i\}_{i \in I}$ is an open cover of $U$, and if for each $i$ we have a section $s_i \in \mathcal{F}(U_i)$ such that for all $i,j \in I$,
\[
s_i \big|_{U_i \cap U_j} 
\;=\; 
s_j \big|_{U_i \cap U_j},
\]
then there is a \emph{unique} section $s \in \mathcal{F}(U)$ such that 
\[
s\big|_{U_i} \;=\; s_i
\quad\text{for each } i \in I.
\]
\end{enumerate}

\medskip

In more intuitive terms: 
\begin{quote}
\textit{The sheaf condition says that if you have local data (sections) that agree on every overlap of the covering sets, then you can ``glue'' them all together to get one global section on the whole set, and this global section is unique once you fix its local restrictions.}
\end{quote}

\subsection*{Diagrammatic Illustration (for two open sets)}

When the open cover consists of two sets $U_1$ and $U_2$ covering $U=U_1 \cup U_2$, the \emph{gluing} condition can be visualized with the following diagram:

\[
\begin{array}{rcccl}
 & & & \mathcal{F}(U) &  \\
 & \swarrow & & \searrow & \\
\mathcal{F}(U_1) & & \longleftrightarrow & & \mathcal{F}(U_2) \\
 & \searrow & & \swarrow & \\
 & & \mathcal{F}(U_1 \cap U_2) & &
\end{array}
\]

Given $s_1 \in \mathcal{F}(U_1)$ and $s_2 \in \mathcal{F}(U_2)$ such that their restrictions agree on $U_1 \cap U_2$, the sheaf condition asserts the existence of a \emph{unique} $s \in \mathcal{F}(U)$ that restricts to $s_1$ on $U_1$ and $s_2$ on $U_2$.

\end{document}
